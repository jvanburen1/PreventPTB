\documentclass[12pt]{article}
% This template is for a general network study, and may be modified for
% other types of study.

% The Table of Contents will automatically adjust for renamed, added,
% or removed headings or subheadings.


\usepackage[headings]{fullpage}
%\usepackage{natbib}
%\bibpunct{(}{)}{;}{a}{,}{,}
\usepackage{graphicx}
\usepackage{multirow}
\usepackage{array}
\usepackage{bm}
\bibliographystyle{plain}

\usepackage{fancyhdr}
\def\protname{Prediction and Prevention of Preterm Birth: A Prospective, Randomized Intervention Trial (PREVENT PTB)}% Type the protocol name here
\def\protacronym{PREVENT PTB }
\def\charterauth{John M. VanBuren, Ph.D.}% Type the SAP author name here
\def\protversion{1.00; October 2, 2017}% Type the protocol version
                                % and date here
\def\charterversion{1.00}% Type the Charter version here
\def\frstverdate{October 2, 2017}% Type date of Charter version 1.0 here
\def\pretrm {PreTRM$^{\small{\textregistered}}$}
\pagestyle{fancy}
\fancyhead{}
\fancyfoot{}
\fancyhead[L]{\protacronym :\ DSMB Charter}
\fancyhead[R]{Page \thepage}
\fancyfoot[L]{\today}
\renewcommand{\footrulewidth}{0.4pt}

\renewcommand{\title}[1]{\begin{center}{\bf\Huge{#1}}\end{center}}
\newcommand{\Comment}[1]{\textbf{[#1]}}

\newcolumntype{P}[1]{>{\centering\arraybackslash}p{#1}}

\begin{document}

\clearpage\pagenumbering{roman}
\thispagestyle{plain}
\title{Data Safety Monitoring Board Charter}

\vspace{1cm}

\begin{center}
\begin{Large}
\textbf{For Intermountain Healthcare Protocol:}
 
\vspace{1cm}

\textbf{\protname}

\end{Large}
\end{center}

\newpage

\begin{center}
DSMB Charter signatures


\textbf{\protname}

\vspace{0.5cm}

Short Title: \protacronym

Phase of Study: III

Lead Investigators and Authors: D. Ware Branch, M.D. and M. Sean Esplin, M.D.

Intermountain Healthcare

\vspace{0.5cm}

Charter Version: \charterversion

Version Date: \today

\end{center}
\vspace{1cm}

\noindent \textit{I confirm that I have read this Charter, I understand it, 
and I will work according to this Charter. I will also work consistently 
with the ethical principles that have their origin in the Declaration of 
Helsinki and are consistent with Good Clinical Practices, applicable Federal 
and state laws, and NIH regulations.}
\vspace{0.5cm}

\noindent Committee Member Name: 

\vspace{0.7cm}
\noindent\rule{16cm}{0.4pt}

\vspace{0.4cm}
\noindent Committee Member Signature:             Date: 

\vspace{0.7cm}
\noindent\rule{16cm}{0.4pt}

\vspace{0.4cm}
\noindent Committee Member Title: 

\vspace{0.7cm}
\noindent\rule{16cm}{0.4pt}

\vspace{0.4cm}
\noindent Committee Member Institution: 

\vspace{0.7cm}
\noindent\rule{16cm}{0.4pt}

\vspace{0.4cm}
\noindent Location of Institution: 

\vspace{0.7cm}
\noindent\rule{16cm}{0.4pt}


\newpage

\tableofcontents

\newpage

\pagenumbering{arabic}

\section{Introduction and Objectives}
The \textit{\protname} is a prospective, multi-center (n=5 sites), 
randomized, intervention trial of screened women compared to unscreened 
controls to assess the effect of the \pretrm \space test. The \pretrm \space 
test is a clinically validated blood test which measures 2 proteins [sex 
hormone binding globulin (SHBG) \& insulin-like growth factor-binding 
protein 4 (IBP4)] that are differentially expressed in mid-pregnancy using 
multiple reaction monitoring mass spectrometry (MRM-MS). 

Women randomized to the intervention group will be screened using the 
\pretrm \space test (Sera Prognostics, Inc.) within an Intermountain 
Healthcare facility or clinic of physicians who deliver patients at an 
Intermountain facility.  Predicated upon the degree of risk, women will be 
offered a treatment according to a pre-specified algorithm with 17-OHPC, low 
dose aspirin, cervical cerclage (if applicable), and high intensity care 
management through the Preterm Prevention Clinic.

The main outcomes will be rates of spontaneous pre-term birth (PTB) in the 
intervention group and the control group. Secondary outcomes include a) 
rates of any PTB in the intervention group and the control group, b) the 
total length of hospital stay for spontaneous preterm births, and c) total 
length of hospital stay for any preterm birth. A blinded adjudication 
committee will review all pre-term deliveries to confirm the categorization 
of spontaneous vs not spontaneous pregnancies. 

The projected accrual period in this study could range between 12 months and 
66 months depending on the enrollment rates and effects observed with an 
anticipated enrollment between 3,000 and 10,000 subjects. The study will 
employ a Bayesian adaptive design allowing for periodic evaluation of the 
data to determine the optimal time to stop enrollment for the study.  Once 
enrollment is stopped, randomized subjects will continue to be followed for 
their entire follow-up until they complete the trial. Trial update analyses 
will assess predicted probabilities of trial success when pre-specified 
enrollment targets have been achieved.  Complete details of the statistical 
design and analysis can be found in the Statistical Analysis Plan.  

The DSMB will monitor safety and the overall study by reviewing aggregate 
and individual subject data related to serious adverse events, data 
integrity (monitoring reports), and overall conduct of the trial.  At 
meetings, the members will recommend whether or not to continue the trial 
based on the totality of the evidence and consideration of the risk/benefit 
ratio.  The DSMB will meet prior to study launch for a kickoff meeting, and 
then approximately every 3 months throughout the study period (i.e., 
approximately 4 meetings per year). The DSMB may also request modifications, 
including additional ad hoc meetings, for safety reasons. The majority of 
the safety meetings will be held via teleconference, but may be held in 
person if agreed upon by the DSMB members as well as Intermountain 
Healthcare. In addition to the safety meetings, the DSMB members will review 
trial updates throughout the trial. 

Throughout this charter, the principal investigators (Drs. Sean Esplin and 
Ware Branch) are referred to as the Sponsor of the study. This Charter 
describes the objectives and procedures to be followed by the DSMB for the 
Prediction and Prevention of Preterm Birth: A Prospective Randomized 
Intervention Trial. The Charter is intended to be a living document. The 
DSMB may wish to review it at regular intervals to determine whether any 
changes in procedure are needed.  


\section{Responsibilities of the DSMB}


\subsection{General Role and Responsibilities}
The role of the DSMB is to act in an independent, advisory capacity to the 
Sponsor, providing guidance to help ensure:

\begin{itemize}
\item The protection of human subjects participating in the study; 
\item The proper conduct of the trial and interpretation of interim and final data; and
\item The ongoing scientific validity, integrity, and clinical and scientific relevance of the study.
\end{itemize}


The DSMB will provide recommendations about starting, continuing, and/or 
stopping the PREVENT PTB study based on considerations of treatment 
efficacy, patient safety, and trial futility as appropriate. In addition, 
the DSMB may make observations or recommendations to the Sponsor about, but 
not limited to, the following:

\begin{itemize}
\item Participant safety;
\item Identification of and responses to serious adverse events and patterns in serious adverse events;
\item Benefit/risk ratio of procedures and participant burden;
\item Selection, recruitment, and retention of participants;
\item Adherence to protocol requirements;
\item Efficacy of the study intervention;
\item Completeness, quality, and analysis of measurements;
\item Amendments to the study protocol and consent forms; and
\item Performance of individual centers and core labs.
\end{itemize}

These recommendations will be provided to the Data Coordinating Center (DCC) 
who will relay the information to the Sponsor, the participating centers and 
their respective Institutional Review Boards/Research Ethics Board(s), and 
regulatory agencies.



\subsection{Role of DSMB in Initiation of Patient Enrollment}
Prior to the initiation of patient enrollment, the DSMB will review the 
study protocol.  Any formal amendments to the approved clinical protocol 
during the conduct of the study will also be reviewed by the DSMB.



\subsection{Role of DSMB in Reviewing and Monitoring Serious Adverse Events (SAEs)}
Serious adverse event data will be reported by investigators, as detailed in 
the protocol. Information regarding related and unexpected SAEs will be 
forwarded to the DSMB chair by the DCC. At each DSMB meeting full 
tabulations of SAEs will be reviewed. 

\subsection{Role of DSMB in Trial Updates}
Trial updates (for the primary efficacy outcome) are pre-specified in the 
protocol and will occur after every 500 subjects are enrolled, starting at 
3000 subjects enrolled. At each trial update, spontaneous preterm birth 
(sPTB) status will be used to compute two predictive probabilities of trial 
success given current enrollment and outcome rates.  The first quantity, 
$PP_{now}$, includes all women enrolled including those not having yet given 
birth.  This entails calculating the predictive probability of the number of 
sPTB expected in the remaining women given the current posterior probability 
of sPTB. The second quantity, $PP_{max}$, includes not just women enrolled 
who have not yet given birth but also includes all women on the study 
horizon out to a maximum of 10,000 total study participants.

Trial update analyses are made after 3000, 3500, 4000, 4500, 5000, 5500, 
6000, 6500, 7000, 7500, 8000, 8500, 9000, and 9500 patients are enrolled.   
If the predictive probability of trial success at the current sample size, 
$PP_{now}$, is greater than or equal to $S_n$ in Table~
\ref{table:ProbabilityTable} below, then accrual stops and all women are 
followed until the end of their pregnancy.  Then the final analysis is 
performed.  This indicates that the current number of enrolled women (if 
followed to the end of the pregnancy) offers high probability of statistical 
significance.

If the predictive probability of trial success at the maximum sample size, 
$PP_{max}$, is less than or equal to $F_n$ in Table~
\ref{table:ProbabilityTable} below, then the trial stops for futility.  This 
indicates that even the maximum sample size is unlikely to offer sufficient 
power to demonstrate a statistically significant effect for the primary 
endpoint. 

Otherwise accrual continues and another trial update analysis is performed 
after an additional 500 patients are enrolled.  The final trial update 
analysis is when 9500 patients are enrolled since the maximum sample size is 
10,000 women. These cutoffs were created through simulation. If the study 
follows the pre-specified statistical boundaries, the final analysis will be 
performed with a one-sided chi-square test with an $\alpha$ level of 0.018.

\begin{center}
\begin{table}[h]
\centering
\begin{tabular}{P{4cm}P{2cm}P{2cm}}
\hline
Patients Accrued At Trial Update Analysis & $S_n$ & $F_n$ \\ \hline
3000 & 0.99 & 0.01 \\
3500 & 0.99 & 0.015 \\
4000 & 0.985 & 0.02 \\
4500 & 0.98 & 0.035 \\
5000 & 0.975 & 0.05 \\
5500 & 0.97 & 0.05 \\
6000 & 0.965 & 0.05 \\
6500 & 0.96 & 0.05 \\
7000 & 0.955 & 0.05 \\
7500 & 0.95 & 0.05 \\
8000 & 0.95 & 0.05 \\
8500 & 0.95 & 0.05 \\
9000 & 0.95 & 0.05 \\
9500 & 0.95 & 0.05 \\ \hline
\end{tabular}
\caption{Trial success and futility thresholds for interim analyses}
\label{table:ProbabilityTable}
\end{table}
\end{center}

The trial updates will be performed by a Statistical Analysis Committee 
(SAC) of independent statisticians and results summarized in a trial update 
report.  The trial update report will include a brief summary of enrollment 
and subject status, and the predictive probability results. The trial update 
report will not be a full summary of trial results and will not include 
safety summaries. 

The DSMB deliberations will be guided by the trial design as defined in the 
study protocol and adaptive design report, although the DSMB may make 
recommendations that deviate from the trial design if necessary to protect 
patient safety, or based on considerations of treatment efficacy, harm, or 
trial futility.  DSMB recommendations will be communicated to the DCC. The 
DCC will disseminate the information to the appropriate parties.  At the 
time of the trial update, the Sponsor may request that the DSMB discuss 
results with representatives from regulatory authorities such as the FDA.


\section{Responsibilities of the Sponsor}
The Sponsor will do the following:
\begin{itemize}
\item Ensure, prior to patient enrollment, the DSMB, SAC, and DCC are familiar with 1) the study design, 2) trial- or program-specific information, and 3) trial update plans.
\item Respect the independence of the DSMB by maintaining communications with the DSMB only through open meetings and the formal recommendation process described in Section 8.
\item Ensure the DSMB, SAC and DCC have sufficient resources to carry out their designated functions, including full access to unmasked accumulating study data and the ability to perform additional analyses requested by the DSMB.
\item Inform the DSMB of all protocol amendments in a timely manner. 
\item Communicate with regulatory authorities, IRB, and investigators, in a manner that maintains integrity (e.g., blinding) of the data, as necessary. (This communication is not the responsibility of the DSMB.)
\item Provide funding for the study and DSMB.
\end{itemize}	

\section{Responsibilities of the Data Coordinating Center}
The Data Coordinating Center is responsible for the following:

\begin{itemize}
\item Ensuring the completeness and accuracy of data collected to the extent required by the DSMB and Sponsor.
\item Providing analysis data sets to the independent SAC performing the trial updates.
	\begin{itemize}
	\item The timing of the trial updates are pre-specified in the protocol.  Database extracts will occur 2 business days after the trial update trigger and analysis datasets provided to the SAC one business day later, i.e. 3 business days after the trial update trigger.
	\end{itemize}
\item Participate in the DSMB kick-off meeting to discuss DSMB report structure, content, and transmission procedures.
\item Prepare full DSMB reports for both the open and closed sessions. Database extracts for the full safety reports will be, at most, 3 weeks before the report is distributed to the DSMB, i.e. 4 weeks before the DSMB meeting.  The open session reports will contain only aggregate/blinded summaries.  The closed session reports will contain summaries by treatment assignment and only be available to unblinded parties.  Reports will be sent to the DSMB 5 business days before the scheduled meetings.
\item Respond to requests for additional analyses from the DSMB.
\item Participate in DSMB meetings to answer questions regarding the report contents or questions regarding the data.
\end{itemize}	

The University of Utah DCC statistical team will be unblinded.  The non-
statistical staff at the DCC will remain blinded during the study.


\section{Responsibilities of the Statistical Analysis Committee Performing the Trial Updates}
The Statistical Analysis Committee (SAC) is the independent statisticians 
performing the trial updates.  This group will be responsible for the 
following:
\begin{itemize}
\item Participate in the DSMB kick-off meeting, present the design, and answer DSMB questions regarding the design.
\item Receive unblinded analysis datasets from the DCC.  The analysis datasets will be a subset of the complete data and limited to the information required for performing and evaluating the trial updates.
\item Perform the trial updates as pre-specified by the protocol and review whether the algorithms are running appropriately and have the appropriate information.
\item Prepare a summary report of the results of each trial update and send to the DSMB.  Trial updates and reports will be created within 5 business days of receiving the data transfer.  
\item Participate in the DSMB meetings to present the trial update results and provide interpretation of results if requested.
\item Respond to any questions from the DSMB regarding the trial updates and perform additional analyses directly related to the trial update analyses if requested by the DSMB.
\end{itemize}

The SAC is a team from Berry Consultants and they will be unblinded.


\section{Structure and Membership of the DSMB}

\underline{Composition}

The DSMB will be composed of a minimum of 4 members. The membership will 
include representation from experts in the fields of maternal fetal medicine 
and biostatistics.

\vspace{0.4cm}
\noindent \underline{Term}

The duration of membership for the DSMB will include all planned analyses in 
the PREVENT PTB study. In the event that a member withdraws from the Board, 
Intermountain Healthcare will appoint a replacement in a timely fashion. 

It is expected that all DSMB members will attend every meeting and 
conference call. However, it is recognized that this may not always be 
possible. For a scheduled meeting, a quorum of this DSMB is considered to be 
3 members, but one attendee must be the DSMB biostatistician. All standing 
DSMB members are voting members. Members who are unable to attend a meeting 
should provide written comments to the Chair based on materials provided one 
week prior to the meeting.  In an extraordinary circumstance in which the 
chair is unable to participate in DSMB deliberations, and an urgent DSMB 
meeting is required to ensure research subject safety, an acting DSMB chair 
may be selected by the DSMB members from the other DSMB members.


\section{Conflict of Interest and Financial Disclosure}
In accordance with the FDA guidance document addressing the structure and 
operation of clinical trial data monitoring committees, DSMB members should 
be independent of entities sponsoring, conducting, or regulating the PREVENT 
PTB study. Specifically, DSMB members should not have any significant 
financial interest in the study's conduct or outcome, (this DSMB will be 
independent of Intermountain Healthcare, Sera Prognostics, Inc., regulatory 
agencies, Institutional Review Board/Research Ethics Board, Berry 
Consultants, and investigators). Additionally, members must disclose any 
actual or potential conflicts of interest involving pharmaceutical or 
biotechnology companies and contract research organizations. This includes 
any financial arrangement, consultancy agreement (direct or via third 
party), research support, or any other relationship that could be construed 
as introducing potential bias to their role as a DSMB member. 

The DSMB will be responsible for determining whether any consultancies or 
financial interests of a member may be viewed as potentially materially 
impacting their objectivity. This decision is to be based on the reasonable 
belief that objectivity is in doubt. 

\section{Confidentiality}
All members will treat as confidential the data, reports, meeting 
discussions, and minutes.

The DSMB members agree to keep completely confidential and not make 
accessible to third parties any confidential information, business secrets 
and other proprietary information furnished by the other party pursuant to 
the establishment of the DSMB. This provision shall be in force during and 
after the termination of the participation in the DSMB. The terms and 
conditions are more particularly set out in a Confidentiality Agreement to 
be signed by each DSMB member. The Confidentiality Agreement is a binding 
part of this Agreement. 

Confidential Information means all information relating to studies with the 
sponsor disclosed by or on behalf of the sponsor, whether disclosed in 
writing, verbally or by any other means and regardless of the date it was 
disclosed.

\section{Communication Plan}
Any communication between the DSMB and study investigators outside of the 
meeting format should be directed through the DSMB Chair and then through 
the DCC statistician. This will facilitate appropriate dissemination of 
information to other DSMB members, promote the DSMB’s role as an objective 
and independent entity, and ensure that study personnel remain blinded to 
the results of the trial update analyses. 

\section{DSMB Meetings}
A meeting of the DSMB may be called at any time by the chair of the DSMB or 
the Sponsor.  If the DSMB chair and the Sponsor are in disagreement 
regarding the need for a DSMB meeting, the opinion of the DSMB chair will 
prevail.

The DSMB meeting plan for this study includes approximately 4 safety 
meetings per year after the initial meeting. The majority of the meetings 
will be held via teleconference, but may be held in person if agreed upon by 
the DSMB members as well as Intermountain Healthcare.

\vspace{0.4cm}
\noindent \underline{Initial Meeting}

The initial meeting of the DSMB will be held prior to the beginning of 
patient enrollment. This meeting will allow DSMB members the opportunity to 
discuss the logistical issues pertaining to future meetings, including data 
provided for review, the proposed format of the meetings, and the handling 
of meeting minutes. The protocol will be reviewed and the design discussed 
to ensure an understanding of the trial updates and how they will be 
conducted.   In addition, this DSMB charter will be discussed and approved. 
Potential conflicts of interest will also be addressed at this time. 

\vspace{0.4cm}
\noindent \underline{Conduct of DSMB Meetings}

\vspace{0.2cm}
\noindent \textit{Creation of the Report}

The University of Utah Data Coordinating Center will be responsible for the 
creation of the full DSMB meeting reports.

\vspace{0.4cm}
\noindent \textit{Open Session} 

This component of the meeting will be open to DSMB members, the DCC members 
as appropriate, the study Principal Investigator, representatives of 
Intermountain Healthcare, representatives of Berry Consultants, and 
representatives of Sera Prognostics. This session will involve review of 
study recruitment data, characteristics of enrolled subjects, and other 
administrative issues not involving disclosure data by treatment arm. During 
open sessions, only non-confidential information that does not threaten the 
integrity or feasibility of the study will be discussed.

\vspace{0.4cm}
\noindent \textit{Closed Session}

This component of the meeting will be open to DSMB members, the DCC 
Biostatistician(s), and representatives of the SAC only. All matters and 
information, impacting the safety, ethics, and scientific validity and 
integrity of the studies may be discussed during closed sessions. All formal 
recommendations considered by the DSMB will be discussed during closed 
sessions.

In the study safety results presented during the closed session, the 
treatment arms will be labeled as Arm A and Arm B for confidentiality 
reasons. The DSMB will be notified in a separate document the true identity 
of Arm A and Arm B.

\vspace{0.4cm}
\noindent \textit{Executive Session (Optional)}

This component of the meeting will be limited to DSMB members, providing the 
opportunity for independent discussion of all aspects of study progress and 
drafting of recommendations. 

\vspace{0.4cm}
\noindent \textit{Debriefing Session (Optional)}

This final component of the meeting will be open to DSMB members, the DCC 
members as appropriate, the study Principal Investigator, representatives of 
Intermountain Healthcare, representatives of Berry Consultants, and 
representatives of Sera Prognostics. The DSMB will provide formal 
recommendations regarding continuation of the study, which will be discussed 
as necessary. 

\vspace{0.4cm}
\noindent \textit{Additional Data Requests}

Prior to a meeting, the DSMB may request additional data items that they 
believe are necessary to facilitate their role in the study. Such requests 
should be made by the DSMB Chair directly to the DCC PI or Biostatistician. 
The DCC will make every reasonable effort to disseminate such additional 
data to the DSMB members prior to the meeting date. The nature of such 
requests will not be communicated to the Sponsor, other investigators, or 
other study personnel, to avoid biasing or impacting study content.


\section{DSMB Recommendations}
The DSMB may make recommendations to the sponsor regarding study 
modifications, continuation, or discontinuation of the study based on 
information available at the time of each meeting. Such information may 
include findings reported outside of this study. A majority of DSMB members 
is required for a recommendation, with the biostatistician being one of 
them. To vote, a Committee member must be present in person or participating 
via conference call. In the event that an issue was not resolved by a 
majority opinion, this should be communicated to the sponsor during 
debriefing.  If all issues were resolved, the DSMB Chair may communicate one 
of the three verbal recommendations to Intermountain Healthcare during the 
debriefing session (if it occurs):

\begin{itemize}
\item Discontinuation of the study
\item Modification of the study protocol or procedures (including, but not limited to, changes in eligibility criteria, regimen of the study intervention, or frequency of visits or site monitoring)
\item Continue the study according to the protocol and any related amendments
\end{itemize}

Within 5 days following the DSMB meeting, the DSMB should write an official 
written recommendation to the DCC. The DCC will then relay the 
recommendation to NAME. The rationale for a DSMB recommendation may or may 
not be given, consistent with maintaining the scientific integrity of the 
study.  The letter will be transmitted via email, in a secure password 
protected file, that requires confirmation of receipt. 

If, in the opinion of the DSMB, rapid communication of information or 
recommendations from the DSMB to trial investigators is required to ensure 
the safety of study participants or the integrity of the trial, then the 
DSMB chair will communicate this to the DCC during or immediately after the 
final open session of a DSMB meeting. Preferably then, but at the latest 
within 3 working days, the DSMB chair will prepare a recommendation for 
communication to trial investigators, which, after review from the NAME, 
will be forwarded to all participating investigators through the DCC. 

It is expected that trial investigators will not communicate with DSMB 
members about the study directly, except when making presentations or 
responding to questions at DSMB meetings or during conference calls.
If the DSMB does not identify any safety or other protocol-related concerns 
during a meeting then, within 10 days after receiving the written summary of 
the DSMB meeting the DCC will prepare a statement or Summary Report for 
distribution to the clinical centers that will state that: 
\begin{itemize}
\item A review of outcome data, adverse events, and information relating to study performance (e.g., data timeliness, completeness, and quality) across all centers took place on a given date; and
\item The DSMB recommended that the study continue without modification of the protocol or informed consent.
\end{itemize}


If concerns are identified, the Summary Report will include the DSMB’s 
recommendation, instructions from the DCC, if any, and the remaining Summary 
Report content will be modified appropriately.  The DSMB will NOT provide 
communication letters directly to the study team.  

The DSMB members’ expertise notwithstanding, the final authority to 
implement or reject DSMB recommendations rests with the DCC which may, if 
appropriate, seek additional input from regulatory authorities or outside 
consultants. The DCC will inform the DSMB of the decision to implement or 
reject recommendations. 

\vspace{0.4cm}
\noindent \textit{Documentation}

It will be the responsibility of the DSMB chairperson to ensure proper 
documentation of all recommendations made by the DSMB. The meeting minutes 
will include the names of attendees at the open and closed sessions, a 
factual summary of the discussion, and a summary of recommendations. The 
meeting minutes for the open and closed sessions will be separated, and 
minutes of the open session will be provided to Intermountain Healthcare and 
the study investigators. The meeting minutes for the closed session will be 
retained by the DCC statistician until the study is formally terminated and 
all statistical analyses have been completed.  Minutes for the closed 
session should include the DSMB rationale and considerations for any 
recommendation.

\vspace{0.4cm}
\noindent \underline{Conduct of Trial Update Reviews}

Trial updates need not coincide with quarterly DSMB meetings and the DSMB is 
not required to hold a full safety review at the time of each trial update.  
The DSMB will be notified by the DCC as a trial update nears based on the 
pre-specified enrollment triggers.  The SAC will provide the trial update 
report directly to the DSMB members. The trial update report will include a 
brief summary of enrollment and subject status, and the predictive 
probability results. The trial update report will not be a full summary of 
trial results and will not include safety summaries.  The report will be 
available 8 business days after the trial update trigger has been reached, 
i.e. 5 business days after the SAC receives the data transfer.   The DSMB 
may decide whether or not to call an abbreviated closed session meeting via 
teleconference to discuss the results.  If a meeting is called, only DSMB 
members, the SAC, and the DCC Biostatistician(s), will attend.
If the timing of trial update aligns with a full DSMB meeting, the trial 
update results will be presented during the closed session of the meeting. 

\vspace{0.4cm}
\noindent \textit{DSMB Communications}

The DSMB will communicate to the DCC regarding whether to continue or stop 
enrollment according to the trial update as pre-specified in the protocol. A 
majority of DSMB members is required for a recommendation, with the 
biostatistician being one of them. To vote, a Committee member must be 
present in person or participating via conference call.  DSMB 
recommendations will be made in writing within 3 business days of receipt  
of the trial update report.  As part of their recommendation, the DSMB Chair 
will communicate one of the three actions to the DCC:
\begin{itemize}
\item Discontinuation of the study due to futility and stop enrollment/follow-up
\item Stop enrollment in study due to predicted success and continue follow-up on all women currently enrolled.
\item Continue the study according to the protocol 
\end{itemize}


If the DSMB decides to deviate from the pre-specified trial design (e.g., 
not make recommendations to the sponsor in a manner that is in accordance 
with the pre-specified adaptation or other decision rules defined by the 
trial design), then the following must occur: 
\begin{itemize}
\item The DSMB must immediately contact the SAC personnel who created the report on which their recommendation is based and explain their rationale for not following the pre-specified decision criteria, and one of the following approaches must be taken to resolve the issue:
\begin{itemize}
\item If the DSMB and SAC agree that a temporary deviation from the pre-specified decision criteria (e.g., while additional data are collected or verified, or additional sensitivity analyses are performed), then:
\begin{itemize}
\item The DSMB and SAC must agree on the steps necessary to address outstanding issues so that the question of following the pre-specified trial design can be reconsidered;
\item The Sponsor does not need to be notified of the deviation while the outstanding issues are being addressed, as doing so might unnecessarily unblind the Sponsor to trial update results; and
\item The DSMB and SAC must agree on the maximum duration of time that can elapse before the Sponsor is informed, with or without resolution of the outstanding issues and, if that time elapses without resolution of the issue then the Sponsor must be notified of the deviation.
\end{itemize}

\item If the DSMB and SAC agree that there should be a permanent deviation from the pre-specified decision criteria (e.g., if it is found that a decision rule is no longer appropriate because of unanticipated patterns in the data or new information, such as safety data) then:
\begin{itemize}
\item This joint decision and the rationale will be recorded in the closed minutes of the DSMB and by the chair of the SAC; and
\item The sponsor does not need to be notified of the deviation until the termination of the trial or a time at which the deviation must be revealed to allow the Sponsor to continue to conduct the trial.
\end{itemize}

\item If the DSMB and SAC disagree on the need to deviate from the pre-specified trial design, then:
\begin{itemize}
\item The DSMB must inform the DCC of the decision or action defined by the pre-specified trial design, their recommendation, and their rationale for deviating from the pre-specified design.
\item The DCC will inform NAME of the disagreement and NAME will make the decision to accept or reject the DSMB recommendation.
\end{itemize} 
\end{itemize}
\end{itemize}


\section{Contents of Report to the DSMB}
Summaries presented to the DSMB in the open session will include screening 
and recruitment numbers to date and overall baseline characteristics of 
randomized study patients. In the closed session, data presented will 
include subject characteristics by assigned treatment, serious adverse event 
rates by assigned treatment, and protocol adherence rates. The proposed 
contents of each of these sections are discussed in additional detail below.

\vspace{0.4cm}
\noindent \textit{Subject Recruitment}

Number of subjects consented, randomized, and completing the study will be 
presented overall and by clinical center. Key reasons for exclusion and 
proportion of eligible subjects randomized by site will be presented. Graphs 
will be shown of recruitment over time, extrapolating to expected study 
numbers if recruitment continues at current rates.

\vspace{0.4cm}
\noindent \textit{Overall Subject Characteristics}

This section will describe medical history, physical examination data, and 
admission information.
This portion of the presentation is intended to foster dialogue between the 
DSMB and PI regarding composition of the study population at study entry, 
and the timing of randomization.

\vspace{0.4cm}
\noindent \textit{Subject Characteristics by Assigned Treatment}

After the meeting moves to closed session, distributions of key subject 
characteristics will be presented according to assigned treatment. This 
portion of the presentation is intended to assess if there is balance 
between the treatment arms with respect to key characteristics.

\vspace{0.4cm}
\noindent \textit{Serious Adverse Events}

Rates of serious adverse events will be presented by treatment arm in the 
closed session report. Events will be summarized by intensity, by suspected 
relationship between study drug(s) and event, by severity, and by 
expectedness. Treatment and action taken and outcome will be presented. When 
appropriate, detailed descriptions of specific events will be presented.  In 
particular, the following summaries will include (but is not limited to):
\begin{itemize}
\item Maternal death
\item Neonatal death
\item Composite neonatal morbidity including
\begin{itemize}
\item NICU admission
\item Bronchopulmonary Dysplasia
\item Intraventricular hemorrhage
\item Retinopathy of prematurity
\item Culture proven sepsis
\item Necrotizing enteritis
\end{itemize}
\end{itemize}

\vspace{0.4cm}
\noindent \textit{Efficacy}

The primary clinical endpoint in this trial (i.e., spontaneous pre-term 
birth) will be referred to the Adjudication Committee (AC) for adjudication. 
The DSMB reports will include adjudicated data whenever possible. If time 
does not allow for AC review prior to a DSMB meeting, partially 
unadjudicated data will be presented, noting such to the DSMB.  Data 
presented will include primary and secondary endpoints by assigned treatment 
(overall and by subgroups). 

\section{DSMB Members}
\begin{tabular}{cc}

Name & Specialty \\ \hline
Name 1 & Maternal Fetal Medicine \\
Name 2 & Maternal Fetal Medicine \\
Name 3 & Maternal Fetal Medicine \\
Name 4 & Biostatistics \\ \hline
\end{tabular}

\end{document}
